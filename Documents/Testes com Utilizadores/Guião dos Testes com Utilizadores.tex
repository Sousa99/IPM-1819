\documentclass[12pt]{article}

\usepackage[a4paper, margin=2.5cm]{geometry}
\usepackage[utf8]{inputenc}
\usepackage{color,soul}
\usepackage{enumitem}
\usepackage[portuguese]{babel}
\usepackage{float}
\usepackage[font=small,labelfont=bf]{caption}
\usepackage{titling}
\renewcommand{\rmdefault}{ptm}

\setlength{\droptitle}{-5em}

\author{
  Isabel Soares\\
  \texttt{89466}
  \and
  Rodrigo Sousa\\
  \texttt{89535}
  \and
  Tiago Afonso\\
  \texttt{89549}
}

\title{Interfaces Pessoa Máquina\\
Guião dos Testes com Utilizadores}
    

\begin{document}
    

\maketitle

\section*{Preparação}
    Os testes com utilizadores irão ser elaborados em salas de aula do IST, num computador de um dos membros do grupo.
    
    \textbf{Para a realização do teste será necessário o seguinte material:}
    Um computador; acesso à Internet para aceder à versão online do protótipo ou uma cópia local \textbf{exata} do protótipo no próprio computador; folha Excel onde será feito o registo da avaliação do protótipo com o utilizador; cronómetro para registo do tempo demorado pelo utilizador nas diversas tarefas( é importante que este cronómetro seja capaz de registar tempos parciais); folha onde o utilizador assinará de forma a dar autorização ao registo e uso de informações recolhidas durante o teste bem como de dados demográficos.

\section*{Introdução}
    \begin{enumerate}
        \item \textbf{Apresentação:}\\
        \textit{Bom dia / Boa tarde: \\
        Somos alunos de 2º ano de LEIC (Licenciatura de Engenharia Informática e de Computadores), e no âmbito da cadeira  IPM (Interfaces Pessoa Máquina) desenvolvemos de uma interface para o iGo, um dispositivo que será usado no braço. Este dispositivo foi concibido com o intuito de ajudar o utilizador durante as suas viagens!\\Foi nos pedido que desenvolvêssemos três funcionalidades à nossa escolha:
        \begin{enumerate}
            \item Partilha de localização e vídeo em tempo real (streaming) com um grupo fechado de amigos criado pelo utilizador.
            \item Criação de uma rota de viagem diária, onde o utlizador pode planear o seu dia
            \item Uma funcionalidade mais direcionada para a saúde nomeadamente SOS, registo de atividades e registo de informações sobre o utilizador como batimento cardiaco, sono, oxigenação do sangue, entre outras.
        \end{enumerate}
        As três funcionalidades serão avaliadas num teste cuja duração não é superior a 5 minutos.\\
        }
        
        \item \textbf{Objetivos do teste:}\\
        \textit{É importante que saiba que durante esta "avaliação" o foco não será em si, que testa o protótipo, mas sim no protótipo em si. Pelo que qualquer problema ou impasse com que se depare será um problema do protótipo que nós deveríamos ter tido em consideração ou prevenido.}
        
        \item \textbf{O que irá ser pedido:}\\
        \textit{Irá ser-lhe pedido que realize três tarefas distintas.\\Iremos explicar-lhe a primeira tarefa, irá executá-la e só de seguida será lhe dito a seguinte de forma a preocupar-se com uma tarefa de cada vez. No final será lhe feito um pequeno questionário sobre o protótipo e a sua interface.\\
        A qualquer momento poderá pedir que lhe seja relembrada a tarefa, mas esta será dita da mesma forma e não será dada qualquer ajuda extra, poderá também desisitir da tarefa pedida!}
        
        \item \textbf{Questionários pré-teste para recolha de dados demográficos:}\\
        \textit{Temos um pequeno questionário pré-teste de forma a recolher alguns dados demográficos:
            \begin{enumerate}
                \item Quais são as suas habilitações literárias?
                \item Já utilizou um dispositivo smartwatch?
                \item Como costuma encontrar/pedir direções?
            \end{enumerate}}
    \end{enumerate} 
\section*{Avaliação}
    \begin{enumerate}
        \item \textbf{Tarefa 1: Adicionar o contacto Maria Correia ao grupo e de seguida fazer streaming de vídeo.}\\
        Medida e Critério de:
        \begin{enumerate}
            \item \textbf{Eficácia:} Quantos cliques foram necessários para o utilizador concluir a tarefa?\\
            \underline{Critério:} Em média os utilizador devem concluir a tarefa com cerca de 14 cliques.
            
            \item \textbf{Eficiência:} Quanto tempo demorou o utilizador a concluir a tarefa?\\
            \underline{Critério:} O utilizador deve concluir a tarefa em 45 segundos ou menos.
            
            \item \textbf{Satisfação:} Como classificaria numa escala de 1 a 5, respetivamente Muito Díficil e Muito Fácil, a tarefa que lhe foi requerida?\\
            \underline{Critério:} Em média os utilizadores devem classificar a tarefa com nível 3 ou superior.
        \end{enumerate}
        
        \item \textbf{Tarefa 2: Criar uma rota de viagem que inclua uma ida à Gulbenkian, jantar no Mc Donald's e ficar alojado no Inatel Oeiras, deslocando-se para todos os locais de metro. No final ver a rota criada no mapa.}\\
        Medida e Critério de:
        \begin{enumerate}
            \item \textbf{Eficácia:} Quantos cliques foram necessários para o utilizador concluir a tarefa?\\
            \underline{Critério:} Em média os utilizador devem concluir a tarefa em 40 ou menos cliques.
            
            \item \textbf{Eficiência:} Quanto tempo demorou o utilizador a concluir a tarefa?\\
            \underline{Critério:} O utilizador deve concluir a tarefa em 90 segundos ou menos.
            
            \item \textbf{Satisfação:} Como classificaria numa escala de 1 a 5, respetivamente Muito Díficil e Muito Fácil, a tarefa que lhe foi requerida?\\
            \underline{Critério:} Em média os utilizadores devem classificar a tarefa com nível 2 ou superior.
        \end{enumerate}
        
        \item \textbf{Tarefa 3: Fazer uma nova medição de \emph{Pressão Arterial}, começar uma atividade de corrida e chamar a emergência.} \\
        Medida e Critério de:
        \begin{enumerate}
            \item \textbf{Eficácia:} Quantos cliques foram necessários para o utilizador concluir a tarefa?\\
            \underline{Critério:} Em média os utilizador devem concluir a tarefa em 25 ou menos cliques.
            
            \item \textbf{Eficiência:} Quanto tempo demorou o utilizador a concluir a tarefa?\\
            \underline{Critério:} O utilizador deve concluir a tarefa em 60 segundos ou menos.
            
            \item \textbf{Satisfação:} Como classificaria numa escala de 1 a 5, respetivamente Muito Díficil e Muito Fácil, a tarefa que lhe foi requerida?\\
            \underline{Critério:} Em média os utilizadores devem classificar a tarefa com nível 4 ou superior.
        \end{enumerate}
    \end{enumerate}

\section*{Balanço}
    \begin{enumerate}
        \item \textbf{Apresentação do questionário de satisfação final}
            \textit{As respostas ao questionário devem ser respondidas numa escala de 1 a 5, cujos valores são Discordo Totalmente, Discordo, Não Concordo nem Discordo, Concordo, Concordo Totalmente.}
             \begin{enumerate}[label=\arabic*)]
                \item Gostaria de usar este dispositivo frequentemente.
                \item Achei as funcionalidades desnecessariamente complexas.
                \item Achei fácil de usar/navegar.
                \item Acho que precisaria de ajuda para voltar a usar as funcionalidades.
                \item Achei que as funcionalidades estão bem integradas.
                \item Achei as funcionalidades demasiado inconsistentes.
                \item Acho que o dispositivo tem um nível de aprendizagem fácil e rápido.
                \item Achei o dispositivo demasiado moroso/incómodo de usar.
                \item Senti-me confiante a navegar pelas funcionalidades do dispositivo.
                \item Acho que é necessário aprender muito antes de se poder usufruir do dispositivo.
            \end{enumerate}
        
        \item \textbf{Comentários finais}\\
        \textit{Relembramos que todas as informações e resultados obtidos durante esta avaliação serão mantidos sobre completo anonimato!}
        
        \item \textbf{Agradecimentos}\\
        \textit{Muito obrigado em nome de todo o grupo pela disponibilidade para a realização desta avaliação!}
    \end{enumerate}

\end{document}