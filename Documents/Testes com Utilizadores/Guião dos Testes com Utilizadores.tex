\documentclass[11pt]{article}

\usepackage[a4paper, margin=2.5cm]{geometry}
\usepackage[utf8]{inputenc}
\usepackage{color,soul}
\usepackage{enumitem}
\usepackage[portuguese]{babel}
\usepackage{float}
\usepackage[font=small,labelfont=bf]{caption}
\usepackage{titling}
\renewcommand{\rmdefault}{ptm}

\setlength{\droptitle}{-5em}

\author{
  Isabel Soares\\
  \texttt{89466}
  \and
  Rodrigo Sousa\\
  \texttt{89535}
  \and
  Tiago Afonso\\
  \texttt{89549}
}

\title{Interfaces Pessoa Máquina\\
Guião dos Testes com Utilizadores}
    

\begin{document}
    

\maketitle

\section*{Preparação}
    Os testes com utilizadores irão ser elaborados em salas de aula do IST, num computador de um dos membros do grupo.
    
    \textbf{Para a realização do teste será necessário o seguinte material:}
    Um computador; acesso à Internet para aceder à versão online do protótipo ou uma cópia local \textbf{exata} do protótipo no próprio computador; folha Excel onde será feito o registo da avaliação do protótipo com o utilizador; cronómetro para registo do tempo demorado pelo utilizador nas diversas tarefas (é importante que este cronómetro seja capaz de registar tempos parciais); folha onde o utilizador assinará de forma a dar autorização ao registo e uso de informações recolhidas durante o teste bem como de dados demográficos.

\section*{Introdução}
    \begin{enumerate}
        \item \textbf{Apresentação:}\\
        \textit{Bom dia / Boa tarde: \\
        Somos alunos de 2º ano de LEIC (Licenciatura de Engenharia Informática e de Computadores), e no âmbito da cadeira  IPM (Interfaces Pessoa Máquina) desenvolvemos de uma interface para o iGo, um dispositivo que será usado no braço. Este dispositivo foi concibido com o intuito de ajudar o utilizador durante as suas viagens!\\Foi nos pedido que desenvolvêssemos três funcionalidades à nossa escolha:
        \begin{enumerate}
            \item Partilha de localização e vídeo em tempo real (streaming) com um grupo fechado de amigos criado pelo utilizador.
            \item Criação de uma rota de viagem diária, onde o utlizador pode planear o seu dia
            \item Uma funcionalidade mais direcionada para a saúde nomeadamente SOS, registo de atividades e registo de informações sobre o utilizador como batimento cardiaco, sono, oxigenação do sangue, entre outras.
        \end{enumerate}
        As três funcionalidades serão avaliadas num teste cuja duração não é superior a 5 minutos.\\
        }
        
        \item \textbf{Objetivos do teste:}\\
        \textit{É importante que saiba que durante esta "avaliação" o foco não será em si, que testa o protótipo, mas sim no protótipo em si. Pelo que qualquer problema ou impasse com que se depare será um problema do protótipo que nós deveríamos ter tido em consideração ou prevenido.}
        
        \item \textbf{O que irá ser pedido:}\\
        \textit{Inicialmente será lhe dada, caso o deseje, a oportunidade de se ambientar ao protótipo e à sua interface, no máximo por 60 segundos.\\Irá ser-lhe pedido que realize três tarefas distintas.\\Iremos explicar-lhe a primeira tarefa, irá executá-la e só de seguida será lhe dito a seguinte de forma a preocupar-se com uma tarefa de cada vez. No final será lhe feito um pequeno questionário sobre o protótipo e a sua interface.\\
        A qualquer momento poderá pedir que lhe seja relembrada a tarefa, mas esta será dita da mesma forma e não será dada qualquer ajuda extra, poderá também desisitir da tarefa pedida!}
        
        \item \textbf{Questionários pré-teste para recolha de dados demográficos:}\\
        \textit{Temos um pequeno questionário pré-teste de forma a recolher alguns dados demográficos:
            \begin{enumerate}
                \item Quais são as suas habilitações literárias?
                \item Já utilizou um dispositivo smartwatch?
                \item Como costuma encontrar/pedir direções?
            \end{enumerate}}
    \end{enumerate} 
    
\section*{Caraterização dos utilizadores}
    Foram realizados testes com 28 utilizadores. Estes testes foram elaborados através do guião realizado no laboratório anterior.
    
    Os utilizadores, de idades compreendidas entre os 13 e os 60 anos, sendo a média 24.96, eram cerca de 61\% do sexo masculino. Na maioria, com habilitações literárias até ao secundário e utilizam o Google Maps para pedir direções.
    Estas informações correspondem ao que foi descrito na AUT, sendo a única diferença, que na AUT tínhamos concluido uma população alvo mais focada na população feminina, enquanto que nos testes existiu uma maior percentagem de homens.
    
\section*{Experiência}
    Os testes foram realizados na maioria em salas da RNL no Técnico, pelo que todos utilizadores que aqui testaram o protótipo eram alunos do Técnico, sendo ainda a maior parte do curso de LEIC.

\section*{Avaliação}
    \begin{enumerate}
        \item \textbf{Tarefa 1: Adicionar o contacto Maria Correia ao grupo e de seguida iniciar streaming de vídeo.}\\
        Medida e Critério de:
        \begin{enumerate}
            \item \textbf{Eficácia:} Quantos cliques foram necessários para o utilizador concluir a tarefa?\\
            \underline{Critério:} Em média os utilizador devem concluir a tarefa com cerca de 19 cliques.
            
            \item \textbf{Eficiência:} Quanto tempo demorou o utilizador a concluir a tarefa?\\
            \underline{Critério:} O utilizador deve concluir a tarefa em 45 segundos ou menos.
            
            \item \textbf{Satisfação:} Como classificaria numa escala de 1 a 5, respetivamente Muito Díficil e Muito Fácil, a tarefa que lhe foi requerida?\\
            \underline{Critério:} Em média os utilizadores devem classificar a tarefa com nível 3 ou superior.
            
            \item \textbf{Análise estatística:} \\
            
\begin{center}
 \begin{tabular}{|c | c c c|} 
 \hline
   & Cliques & Tempo & Satisfação \\ [0.5ex] 
 \hline
 Média & 19,21 & 50,32 & 4,07 \\ 
 \hline
 Desvio Padrão & 7,50 & 20,76 & 0,81 \\
 \hline
 Intervalo de Confiança 95\% & $[16,30;22,12]$ & $[42,27;58,37]$ & $[3,76;4,39]$ \\

 \hline
\end{tabular}
\end{center}
            
            \item \textbf{Discussão:} Os critérios de eficácia e satisfação foram cumpridos com sucesso, no entanto o critério de eficiência não foi cumprido.
            Isto deveu-se ao facto dos utilizadores, na maior parte, associarem, corretamente, o streaming de video ao grupo, mas assumindo que se encontraria neste ecrã ou no dos contactos a funcionalidade de streaming de vídeo.
        \end{enumerate}
        
        \item \textbf{Tarefa 2: Criar uma rota de viagem que inclua uma ida à Gulbenkian, jantar no Mc Donald's e ficar alojado no Inatel Oeiras, deslocando-se para todos os locais de metro. No final ver a rota criada no mapa.}\\
        Medida e Critério de:
        \begin{enumerate}
            \item \textbf{Eficácia:} Quantos cliques foram necessários para o utilizador concluir a tarefa?\\
            \underline{Critério:} Em média os utilizador devem concluir a tarefa em 42 ou menos cliques.
            
            \item \textbf{Eficiência:} Quanto tempo demorou o utilizador a concluir a tarefa?\\
            \underline{Critério:} O utilizador deve concluir a tarefa em 100 segundos ou menos.
            
            \item \textbf{Satisfação:} Como classificaria numa escala de 1 a 5, respetivamente Muito Díficil e Muito Fácil, a tarefa que lhe foi requerida?\\
            \underline{Critério:} Em média os utilizadores devem classificar a tarefa com nível 2 ou superior.
            
            \item \textbf{Análise estatística:} \\
\begin{center}
 \begin{tabular}{|c | c c c|} 
 \hline
   & Cliques & Tempo & Satisfação \\ [0.5ex] 
 \hline
 Média & 40,93 & 96,21 & 3,68 \\ 
 \hline
 Desvio Padrão & 10,46 & 46,07 & 1,09 \\
 \hline
 Intervalo de Confiança 95\% & $[36,87;44,99]$ & $[78,35;114,08]$ & $[3,26;4,10]$ \\

 \hline
\end{tabular}
\end{center}
            
            \item \textbf{Discussão:} Todos os critérios foram cumpridos com sucesso.
            Apesar de os utilizadores se depararem com alguns problemas, no entanto, ao contrário até ao esperado, conseguiram resolver os problemas de forma eficaz e até eficientemente.
            Os principais problemas identificados pelos utilizadores foram: associação não imediata do botão que permitia escolha dos locais em modo lista; alguns problemas em identificar o botão de escolha do meio de transporte para o local, muitos só o associaram depois de já terem adicionado um local uma primeira vez.
        \end{enumerate}
        
        \item \textbf{Tarefa 3: Fazer uma nova medição de \emph{Pressão Arterial}, começar uma atividade de corrida e chamar a emergência.} \\
        Medida e Critério de:
        \begin{enumerate}
            \item \textbf{Eficácia:} Quantos cliques foram necessários para o utilizador concluir a tarefa?\\
            \underline{Critério:} Em média os utilizador devem concluir a tarefa em 25 ou menos cliques.
            
            \item \textbf{Eficiência:} Quanto tempo demorou o utilizador a concluir a tarefa?\\
            \underline{Critério:} O utilizador deve concluir a tarefa em 60 segundos ou menos.
            
            \item \textbf{Satisfação:} Como classificaria numa escala de 1 a 5, respetivamente Muito Díficil e Muito Fácil, a tarefa que lhe foi requerida?\\
            \underline{Critério:} Em média os utilizadores devem classificar a tarefa com nível 4 ou superior.
            
            \item \textbf{Análise estatística:} \\
\begin{center}
 \begin{tabular}{|c | c c c|} 
 \hline
   & Cliques & Tempo & Satisfação \\ [0.5ex] 
 \hline
 Média & 23,39 & 53,18 & 4,79 \\ 
 \hline
 Desvio Padrão & 3,95 & 14,78 & 0,42 \\
 \hline
 Intervalo de Confiança 95\% & $[21,89;24,92]$ & $[47,45;58,91]$ & $[4,62;4,95]$ \\

 \hline
\end{tabular}
\end{center}
            
            \item \textbf{Discussão:} Todos os critérios foram cumpridos com sucesso.
            Os utilizadores acharam uma tarefa simples e por isso não sentiram dificuldades na mesma. No entanto alguns assumiram após feita a medição de pressão arterial que o dispositivo mudaria para o ecrã anterior automaticamente ou clicando no símbolo levaria a este, caso houvesse uma fase de melhorar o protótipo após esta avaliação seria algo que se deveria ter em conta.
        \end{enumerate}
    \end{enumerate}

\section*{Balanço}
    \begin{enumerate}
        \item \textbf{Apresentação do questionário de satisfação final}
            \textit{As respostas ao questionário devem ser respondidas numa escala de 1 a 5, cujos valores são Discordo Totalmente, Discordo, Não Concordo nem Discordo, Concordo, Concordo Totalmente.}
             \begin{enumerate}[label=\arabic*)]
                \item Gostaria de usar este dispositivo frequentemente.
                \item Achei as funcionalidades desnecessariamente complexas.
                \item Achei fácil de usar/navegar.
                \item Acho que precisaria de ajuda para voltar a usar as funcionalidades.
                \item Achei que as funcionalidades estão bem integradas.
                \item Achei as funcionalidades demasiado inconsistentes.
                \item Acho que o dispositivo tem um nível de aprendizagem fácil e rápido.
                \item Achei o dispositivo demasiado moroso/incómodo de usar.
                \item Senti-me confiante a navegar pelas funcionalidades do dispositivo.
                \item Acho que é necessário aprender muito antes de se poder usufruir do dispositivo.
            \end{enumerate}
        
        \item \textbf{Análise dos Resultados:}\\
\begin{center}
 \begin{tabular}{|c | c c c |} 
 \hline
 Perguntas & Média & Desvio Padrão & Intervalo de Confiança 95\% \\ [0.5ex] 
 \hline
 1 & 3,89 & 1,13 & $[3,45;4,33]$\\ 
 \hline
 2 & 1,68 & 0,82 & $[1,36;2,00]$\\
 \hline
 3 & 4,14 & 0,52 & $[3,94;4,35]$\\
 \hline
 4 & 1,93 & 0,52 & $[3,94;4,35]$\\
 \hline
 5 & 4,50 & 0,64 & $[4,25;4,75]$\\
 \hline
 6 & 1,43 & 0,63 &$[1,18;1,67]$\\
 \hline
 7 & 4,29 & 0,71 & $[4,01;4,56]$\\
 \hline
 8 & 1,32 & 0,67 & $[1,06;1,58]$\\
 \hline
 9 & 4,25 & 0,89 & $[3,91;4,59]$\\
 \hline
 10 & 1,57 & 0,57 & $[1,35;1,79]$\\
 \hline
\end{tabular}
\end{center}
        
        
        \item \textbf{Discussão Global:}\\
        Observámos que todas as tarefas  atingiram os objetivos com sucesso, apesar da primeria não SE ter cumprido o parâmetro da eficácia, como já foi referido anteriormente.
        Concluímos também que apesar da segunda tarefa ser a mais complexa e o botão de formato de lista não estar assim tão visível para os utilizadores, todos concluíram a mesma com sucesso com mais ou menos tempo.
        
        Numa fase seguinte do desenvolvimento do produto poderiam e deveriam ser realizadas algumas mudanças, embora pequenas, necessárias. Como por exemplo, uma escolha de um botão mais apropriado e evidente do formato de lista no mapa, bem como os botões para escolha do meio de transporte e altura do dia. Por fim, a possibilidade de um atalho no ecrã do grupo que permiti-se levar instantaneamente para a funcionalidade de streaming para o mesmo.
        
        É importante notar que as respostas dos utilizadores ao questionário de satisfação tiveram um balanço extremamente positivo, aproximando-se sempre do valor do ideal para cada uma das questões.
        
        \item \textbf{Comentários finais}\\
        \textit{Relembramos que todas as informações e resultados obtidos durante esta avaliação serão mantidos sobre completo anonimato!}
        
        \item \textbf{Agradecimentos}\\
        \textit{Muito obrigado em nome de todo o grupo pela disponibilidade para a realização desta avaliação!}
    \end{enumerate}
    
\end{document}